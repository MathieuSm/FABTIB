%%%%%%%%%%%%%%%%%%%%%%%%%%%%%%%%%%%%%%%%%
% Beamer Presentation
% LaTeX Template
% Version 1.0 (01/07/19)
%
%%%%%%%%%%%%%%%%%%%%%%%%%%%%%%%%%%%%%%%%%

%----------------------------------------------------------------
%	PACKAGES AND THEMES		-----------------------------------
%----------------------------------------------------------------

\documentclass[xcolor=table,11pt]{beamer}

\mode<presentation> {

\usetheme{Frankfurt}
\usecolortheme{dove}
\usefonttheme{serif}

}
\usepackage{newtxtext,newtxmath}
\usepackage{graphicx}
\usepackage{booktabs} 
\usepackage{subfig}
\usepackage{pgf}
\usepackage{multirow}
\usepackage{appendixnumberbeamer}
\usepackage{bookmark}
\usepackage{siunitx}
\usepackage{animate}
\usepackage{xcolor}
\usepackage{soul}
\usepackage{pifont}
\usepackage{caption}
\captionsetup{skip=0pt,belowskip=0pt}


%----------------------------------------------------------------
%	GENERAL OPTIONS 	-----------------------------------------
%----------------------------------------------------------------

% Set template options
\setbeamertemplate{section in toc}{\inserttocsectionnumber.~\inserttocsection}
\setbeamertemplate{frametitle}{\vspace*{1em}\insertframetitle}
\setbeamertemplate{enumerate items}[default]
\setbeamercolor{section in head/foot}{fg=white, bg=black}

% Headline
\makeatletter
\setbeamertemplate{headline}
{%
  \pgfuseshading{beamer@barshade}%
    \vskip-5ex%
  \begin{beamercolorbox}[ignorebg,ht=2.25ex,dp=3.75ex]{section in head/foot}
  \insertsectionnavigationhorizontal{\paperwidth}{\hskip0pt plus1fill}{\hskip0pt plus1fill}
  \end{beamercolorbox}%
  \ifbeamer@sb@subsection%
    \begin{beamercolorbox}[ignorebg,ht=2.125ex,dp=1.125ex,%
      leftskip=.3cm,rightskip=.3cm plus1fil]{subsection in head/foot}
      \usebeamerfont{subsection in head/foot}\insertsubsectionhead
    \end{beamercolorbox}%
  \fi%
}%
\makeatother

% Footline
\makeatletter
\setbeamertemplate{footline}
{
  \leavevmode%
  \hbox{%
  \begin{beamercolorbox}[wd=.333333\paperwidth,ht=2.25ex,dp=1ex,left]{section in head/foot}%
    \usebeamerfont{author in head/foot}\hspace{10pt}\insertshortauthor
  \end{beamercolorbox}%
  \begin{beamercolorbox}[wd=.333333\paperwidth,ht=2.25ex,dp=1ex,center]{section in head/foot}%
    \usebeamerfont{title in head/foot}\insertshorttitle
  \end{beamercolorbox}%
  \begin{beamercolorbox}[wd=.333333\paperwidth,ht=2.25ex,dp=1ex,right]{section in head/foot}%
    \usebeamerfont{date in head/foot}\insertshortdate{}\hspace*{2em}
    \insertframenumber{}\hspace*{2em}
  \end{beamercolorbox}}%
  \vskip0pt%
}
\makeatother

% Add logo
\logo{\pgfputat{\pgfxy(0,7)}{\includegraphics[width=0.1\paperwidth]{Pictures/00_Unibe_Logo}}}

% Table settings
\renewcommand{\arraystretch}{2}
\captionsetup{labelformat=empty,labelsep=none}
\definecolor{Gray}{gray}{0.9}

% Define highlitghing command
\makeatletter
\let\HL\hl
\renewcommand\hl{%
	\let\set@color\beamerorig@set@color
	\let\reset@color\beamerorig@reset@color
	\HL}
\makeatother

% Add overview at each begin of section
%\AtBeginSection[]
%{
%	\begin{frame}
%		\frametitle{Overview}
%		\tableofcontents[currentsection]
%	\end{frame}
%}


\renewcommand{\arraystretch}{1.2}

%----------------------------------------------------------------
%	TITLE PAGE 	-------------------------------------------------
%----------------------------------------------------------------

\title[Fabric-Elasticity Relationships in OI]{
\uppercase{Fabric-Elasticity Relationships of Tibial Trabecular Bone are Similar in Osteogenesis Imperfecta and Healthy Individuals}
} 

\author[Simon et al.]{Simon M.\inst{a} \and Indermaur M.\inst{a} \and Schenk D.\inst{a} \and Mahdi T.\inst{b,c} \and Willie B. M.\inst{b,c} \and Zysset P.\inst{a}}
\institute{\inst{a}ARTORG Center of Biomedical Engineering Research\\University of Bern \and \inst{b}Departement of Pediatric Surgery\\McGill University, Montreal \and \inst{c}Research Centre\\Shriners Hospital for Children, Montreal
\medskip
}
\date{June 8, 2021}


\begin{document}

\begin{frame}
\titlepage
\end{frame}

%----------------------------------------------------------------
%----------------------------------------------------------------
%----------------------------------------------------------------

\begin{frame}
	\frametitle{Overview}
	\tableofcontents
\end{frame}

%----------------------------------------------------------------
%----------------------------------------------------------------
%----------------------------------------------------------------

\section{Introdution}
\begin{frame}
	\frametitle{Osteogenesis Imperfecta (OI)}
	Metabolic bone disorder
	\begin{itemize}
		\item Prevalence \textasciitilde 1/13'500 birth
		\item Mutations in genes encoding collagen type I
		\item Different Types: I, II, III, and IV
		\item Increased bone fragility $\rightarrow$ \textit{Brittle bone disease}
		\item Blue sclera, hearing loss
	\end{itemize}
	\vfill
	Treatment
	\begin{itemize}
		\item Gold standard: bisphosphonate
		\item Nailing
		\item Physiotherapy
	\end{itemize}
\end{frame}


\begin{frame}
	\frametitle{Study Objective}
	\fontsize{10pt}{10pt}\selectfont
	HR-pQCT scans...
	\begin{itemize}
		\item Distal skeleton \textit{in vivo} assessment
		\item Microstructure analysis
	\end{itemize}\vfill
	Homogenized FE (hFE)...
	\begin{itemize}
		\item Bone strength estimation based on HR-pQCT scans
		\item High correlations between hFE and experiment
	\end{itemize}\vfill
	But...
	\begin{itemize}
		\item HR-pQCT-based FEA relies on fabric-elasticity relationships
	\end{itemize}\vfill
	Therefore...
	\begin{itemize}
		\item Compare trabecular bone microstructure
		\item Investigate fabric-elasticity relationships in OI
	\end{itemize}
\end{frame}


%----------------------------------------------------------------
%----------------------------------------------------------------
%----------------------------------------------------------------


\section{Methods}
\subsection{HR-pQCT Scans}
\begin{frame}
	\frametitle{HR-pQCT Scans}
	\begin{columns}
		\column{.35\textwidth}
		Healthy group:
		\begin{itemize}
			\item 120 participants
			\item In-house protocol
		\end{itemize}\vspace{1em}
		
		OI group:
		\begin{itemize}
			\item 50 participants
			\item Standard protocol
		\end{itemize}
		
		\column{.2\textwidth}
		\centering		
		\includegraphics[width=1\linewidth,trim=200 22 200 0]
		{Pictures/01_ControlClinicalSection2}\\
		Healthy
		
		\column{.2\textwidth}
		\centering
		\includegraphics[width=1\linewidth,trim=200 0 200 22]
		{Pictures/01_OIClinicalSection2}\\
		OI
		
	\end{columns}
\end{frame}

%----------------------------------------------------------------

\subsection{Image Analysis}
\begin{frame}
	\frametitle{Image Analysis}
	\begin{columns}
		\column{.5\textwidth}
		Region of Interest (ROI)
		\begin{itemize}
			\item 6 cubes of 5.3mm
		\end{itemize}\vspace{1em}
		Morphometric analysis
		\begin{itemize}
			\item Trabecular morphology\\
				  Bone volume fraction (BV/TV)\\
			\item Fabric tensor $\mathbf{M}$\\
				  Degree of anisotropy (DA)
			\item Coefficient of variation (CV)
		\end{itemize}
	
		\column{.4\textwidth}
		\centering
		\includegraphics[width=1\linewidth,trim=100 0 100 0]
		{Pictures/01_FabricExample}\\
		Fabric tensor
	\end{columns}
\end{frame}



\begin{frame}
	\frametitle{Fit to Model}
	\begin{columns}
		\column{.5\textwidth}
		Stiffness tensor $\mathbb{S}$
		\begin{itemize}
			\item \si{\micro}FE homogenization
			\item Transformation and projection
		\end{itemize}\vspace{1em}
		Multiple Linear Regression
		\begin{itemize}
			\item Fit to Zysset-Curnier model\\
			$\mathbb{S}$ \textasciitilde \textcolor{blue}{$\lambda_{0}$}, \textcolor{blue}{$\lambda_{0}$'}, \textcolor{blue}{$\mu_{0}$}, $m_{i}^{\textcolor{red}{k}}m_{j}^{\textcolor{red}{k}}$, BV/TV$^{\textcolor{red}{l}}$
			\item Log space for \textcolor{red}{$k$} and \textcolor{red}{$l$}
			\item $R_{adj}^{2}$ and NE for fit quality
		\end{itemize}
		
		\column{.4\textwidth}
		\centering
		\includegraphics[width=1\linewidth,trim=100 0 100 0]
		{Pictures/01_StiffnessExample}\\
		Stiffness tensor
	\end{columns}
\end{frame}



\begin{frame}
	\frametitle{Healthy vs OI Comparison}
	\begin{columns}[t]
		\column{.5\textwidth}
		Morphometric analysis
		\begin{itemize}
			\item Age and gender matching
		\end{itemize}
	
		\vspace{17.5pt}
	
		Tissue BMD (tBMD)
		\begin{itemize}
			\item BV/TV and DA matching
		\end{itemize}
	
		\column{.5\textwidth}
		Linear Regression
		\begin{itemize}
			\item Fit using original data sets
			\item CV filtering
			\item Fit using filtered data sets
			\item BV/TV and DA matching
		\end{itemize}
	\end{columns}

	\vfill

	\begin{table}[b]
		\centering
		\caption{Summary of the data set used for different comparison\vspace{-1em}}
		\label{Table1}\resizebox{1.\linewidth}{!}{\begin{tabular}{p{0.15\linewidth}*{2}{>{\centering\arraybackslash}p{0.1\linewidth}}*{2}{>{\centering\arraybackslash}p{0.15\linewidth}}*{2}{>{\centering\arraybackslash}p{0.1\linewidth}}*{2}{>{\centering\arraybackslash}p{0.15\linewidth}}}
				\toprule
				Data sets & \multicolumn{2}{c}{Original} & \multicolumn{2}{c}{Age \& gender matched} & \multicolumn{2}{c}{CV filtered} & \multicolumn{2}{c}{BV/TV \& DA matched} \\
				\midrule
				Group & Healthy & OI & Healthy & OI & Healthy & OI & Healthy & OI \\
				Individuals & 120 & 49 & 28 & 28 & 119 & 38 & 57 & 32 \\
				ROIs & 720 & 294 & 168 & 168 & 603 & 117 & 82 & 82 \\
				\midrule
				Methods & \multicolumn{2}{c}{Linear regression} & \multicolumn{2}{c}{Morphometric} & \multicolumn{2}{c}{Linear regression} & \multicolumn{2}{c}{Linear regression} \\
				\bottomrule
		\end{tabular}}
	\end{table}

\end{frame}

%----------------------------------------------------------------
%----------------------------------------------------------------
%----------------------------------------------------------------

\section{Results and Discussion}
\subsection{Morphological Analysis}
\begin{frame}
	\frametitle{Morphometric Analysis}
	\begin{table}[]
		\centering
		\caption{Summary of morphometric analysis\vspace{-1em}}
		\label{Table2}\resizebox{1.\linewidth}{!}{
		\begin{tabular}{lcccccccc}
				\toprule
				Variable & BV/TV & Tb.N. & Tb.Th. & Tb.Sp. & Tb.Sp.SD & SMI & DA & Ln(CV) \\
				Healthy ? OI & > & > & = & < & < & < & < & < \\
				p value & <0.01 & <0.001 & 0.2 & 0.01 & 0.02 & <0.001 & 0.02 & <0.0001 \\
				\bottomrule
		\end{tabular}}
	\end{table}\vfill
	Results in agreement with:
	\begin{itemize}
		\item \cite{p9}
		\item \cite{p10}
		\item \cite{p11}
	\end{itemize}
\end{frame}

%----------------------------------------------------------------

\subsection{Linear Regression and Filtering}
\begin{frame}
	\frametitle{Linear Regression - Original Data Sets}
	\begin{itemize}
		\item Healthy results in the expected range
		\item OI fit overestimation for low $\mathbb{S}$ components (low BV/TV)
	\end{itemize}
	\begin{columns}
		\column{0.5\textwidth}
		\centering
		\includegraphics[width=1.\linewidth]{Pictures/02_GR_Healthy_LMM}\\
		Healthy group
		\column{0.5\textwidth}
		\centering
		\includegraphics[width=1.\linewidth]{Pictures/02_GR_OI_LMM}\\
		OI group
	\end{columns}	
\end{frame}



\begin{frame}
	\frametitle{Linear Regression - CV Filtering}
	\fontsize{10pt}{10pt}\selectfont
	\begin{columns}
		\column{0.5\textwidth}
		Homogeneity assumption
		\begin{itemize}
			\item High CV decreases fit quality
		\end{itemize}
	
		\vspace{10pt}
		
		CV is related to BV/TV
		\begin{itemize}
			\item Remove high CV ROIs
		\end{itemize}
	
		\vspace{10pt}
		
		Remove high CV ROIs
		\begin{itemize}
			\item Improve fit quality
			\item Increase results accuracy
		\end{itemize}
		
		
		\column{0.5\textwidth}
		\centering
		\includegraphics[width=1.\linewidth]{Pictures/03_CV_BVTV_ROI}\\
		CV in relation to BV/TV
	\end{columns}	
\end{frame}



\begin{frame}
	\frametitle{Linear Regression - Filtered Data Sets}
	\begin{itemize}
		\item \textasciitilde 1/6 healthy ROIs removed $\rightarrow$ Results in the same range
		\item \textasciitilde 6/10 OI ROIs removed \hspace{0.52cm}$\rightarrow$ Results similar to healthy group
	\end{itemize}
	\begin{columns}
		\column{0.5\textwidth}
		\centering
		\includegraphics[width=1.\linewidth]{Pictures/04_FR_Healthy_LMM}\\
		Healthy group
		\column{0.5\textwidth}
		\centering
		\includegraphics[width=1.\linewidth]{Pictures/04_FR_OI_LMM}\\
		OI group
	\end{columns}	
\end{frame}

%----------------------------------------------------------------

\subsection{BV/TV and DA Matching}

\begin{frame}
	\frametitle{Linear Regression - BV/TV and DA Matching}
	
	Regression conditions:
	\begin{itemize}
		\item Matching allow to fit over same range
		\item Identical $k$ and $l$ for stiffness comparison
	\end{itemize}

	\vfill

	\begin{table}[]
		\centering
		\caption{Summary of BV/TV and DA matched data set linear regression\vspace{-1em}}
		\label{Table3}\resizebox{1.\linewidth}{!}{
		\begin{tabular}{cccccccc}
				\toprule
				Data set & $\lambda_0$ & $\lambda_0'$ & $\mu_0$ & $k$ & $l$ & $R^2_{adj}$ & NE (\%) \\
				\midrule
				Grouped & 4626 [3892-5494] & 2695 [2472-2937] & 3541 [3246-3862] & 1.91 [1.86-1.95] & 0.95 [0.93-0.97] & 0.936 & 19 $\pm$ 9\\
				
				Healthy & 4318 [3844-4851] & 2685 [2533-2845] & 3512 [3306-3731] & \textcolor{gray}{1.91} & \textcolor{gray}{0.95} & 0.835 & 21 $\pm$ 10\\
				
				OI & 4983 [4345-5716] & 2727 [2547-2921] & 3600 [3355-3863] & \textcolor{gray}{1.91} & \textcolor{gray}{0.95} & 0.860 & 20 $\pm$ 10\\
				\bottomrule
		\end{tabular}}
	\end{table}

	\vfill
	
	$\rightarrow$ Overlapping stiffness values: Similar fabric-elasticity relationships
\end{frame}


\begin{frame}[noframenumbering]
	\frametitle{Linear Regression - BV/TV and DA Matching}
	
	Regression conditions:
	\begin{itemize}
		\item Matching allow to fit over same range
		\item Identical $k$ and $l$ for stiffness comparison
	\end{itemize}
	
	\vfill
	
	\begin{table}[]
		\centering
		\caption{Summary of BV/TV and DA matched data set linear regression\vspace{-1em}}
		\label{Table4}\resizebox{1.\linewidth}{!}{
			\begin{tabular}{cccccccc}
				\toprule
				Data set & $\lambda_0$ & $\lambda_0'$ & $\mu_0$ & $k$ & $l$ & $R^2_{adj}$ & NE (\%) \\
				\midrule
				Grouped & 4626 [3892-5494] & 2695 [2472-2937] & 3541 [3246-3862] & 1.91 [1.86-1.95] & 0.95 [0.93-0.97] & 0.936 & 19 $\pm$ 9\\
				
				Healthy & 4318 [\hl{3844-4851}] & 2685 [\hl{2533-2845}] & 3512 [\hl{3306-3731}] & \textcolor{gray}{1.91} & \textcolor{gray}{0.95} & 0.835 & 21 $\pm$ 10\\
				
				OI & 4983 [\hl{4345-5716}] & 2727 [\hl{2547-2921}] & 3600 [\hl{3355-3863}] & \textcolor{gray}{1.91} & \textcolor{gray}{0.95} & 0.860 & 20 $\pm$ 10\\
				\bottomrule
		\end{tabular}}
	\end{table}
	
	\vfill
	
	$\rightarrow$ Overlapping stiffness values: Similar fabric-elasticity relationships
\end{frame}



\begin{frame}
	\frametitle{BV/TV and DA Matching - tBMD}
	\fontsize{10pt}{10pt}\selectfont
	\begin{columns}
		\column{0.5\textwidth}
				
		Results
		\begin{itemize}
			\item tBMD OI > Healthy
			\item tBMD related to BV/TV\\
			$\rightarrow$ Partial volume effect
		\end{itemize}
	
		\vspace{10pt}

		\cite{p12} tBMD
		\begin{itemize}
			\item Modulus
			\item Ultimate stress
			\item Post-yield behavior
		\end{itemize}
		
		\vspace{10pt}
		
		$\rightarrow$ Use tBMD in hFE
		
		\column{0.5\textwidth}
		\centering
		\includegraphics[width=1.\linewidth]{Pictures/05_tBMDvsBVTV}\\
		tBMD in relation to BV/TV
	\end{columns}
\end{frame}

%----------------------------------------------------------------

%\subsection{Comparison with Other Resolutions}
%
%\begin{frame}
%	\frametitle{Comparison with Other Resolutions}
%	\fontsize{10pt}{10pt}\selectfont
%	
%	\begin{table}[]
%		\centering
%		\caption{Summary of studies compared\vspace{-1em}}
%		\label{Table4}\resizebox{1.\linewidth}{!}{
%			\begin{tabular}{lccc}
%				\toprule
%				Study & Scanned site & Scan & Resolution \\
%				\midrule
%				
%				\cite{p13} & Multiple sites & \si{\micro}CT & 18\si{\micro}m $\rightarrow$ 36\si{\micro}m \\
%				
%				\cite{p7} & Proximal femur & \si{\micro}CT & 18\si{\micro}m $\rightarrow$ 36\si{\micro}m \\
%				
%				Present study & Distal tibia & HR-pQCT & 61\si{\micro}m \\
%				\bottomrule
%		\end{tabular}}
%	\end{table}
%
%	\vfill
%
%	\begin{columns}[t]
%		\column{0.5\textwidth}
%		Higher DA
%		\begin{itemize}
%			\item Distal tibia vs proximal femur
%		\end{itemize}
%	
%		\vspace{7pt}
%		
%		Lower stiffness constants ($\lambda_0$, $\lambda_0'$, $\mu_0$)
%		\begin{itemize}
%			\item 61\si{\micro}m vs 36\si{\micro}m
%		\end{itemize}
%	
%		\vspace{7pt}
%		
%		Lower fit quality ($R^2_{adj}$ and NE)
%		\begin{itemize}
%			\item 61\si{\micro}m vs 36\si{\micro}m
%		\end{itemize}
%	
%		\column{0.5\textwidth}
%		Higher $k$
%		\begin{itemize}
%			\item Higher weight of BV/TV
%		\end{itemize}
%	
%		\vspace{7pt}
%		
%		Similar $l$
%		\begin{itemize}
%			\item Similar weight of DA
%		\end{itemize}
%	
%	\end{columns}
%\end{frame}

%----------------------------------------------------------------
%----------------------------------------------------------------
%----------------------------------------------------------------

\section{Conclusions}
\begin{frame}
	\frametitle{Conclusions}
	
	Bone samples...
	\begin{itemize}
		\item Similar as reported in literature
	\end{itemize}
	
	\vfill
	
	Fabric-elasticity relationships...
	\begin{itemize}
		\item Similar in OI and healthy tibial trabecular bone
	\end{itemize}
	
	\vfill
	
	But tBMD higher in OI...
	\begin{itemize}
		\item \cite{p12} $\rightarrow$ use tBMD in hFE
	\end{itemize}
	
	\vfill
	
	Moreover high CV in OI...
	\begin{itemize}
		\item Homogenization process should be adapted
	\end{itemize}
	
\end{frame}


\begin{frame}
	Thank you for your attention !	
\end{frame}

%----------------------------------------------------------------
%----------------------------------------------------------------
%----------------------------------------------------------------
\appendix

\section{References}
\begin{frame}
	\frametitle{Osteogenesis Imperfecta}
	\footnotesize{
		\begin{thebibliography}{99}
			\setbeamertemplate{bibliography item}[triangle]
			
			\bibitem[Mortier et al. 2019]{p1} Mortier et al. (2019)
			\newblock Nosology and classification of genetic skeletal disorders: 2019 revision
			\newblock \textit{American Journal of Medical Genetics, Part A} (179), 2393-2419
			
			\bibitem[Tournis 2018]{p2} Tournis, S., Dede, A.D. (2018)
			\newblock Osteogenesis imperfecta A clinical update
			\newblock \textit{Metabolism: Clinical and Experimental} (80), 27-37
			
			\bibitem[Lim et al. 2017]{p3}  Lim, J., Grafe, I., Alexander, S., Lee, B. (2017)
			\newblock Genetic causes and mechanisms of osteogenesis imperfecta
			\newblock \textit{Bone} (102), 40-49
			
		\end{thebibliography}
	}
\end{frame}



\begin{frame}
	\frametitle{HR-pQCT Scans Protocols}
	\footnotesize{
		\begin{thebibliography}{99}
			\setbeamertemplate{bibliography item}[triangle]
			
			\bibitem[Schenk et al. 2020]{p4}  Schenk, D., Mathis, A., Lippuner, K., Zysset, P. (2020)
			\newblock In vivo repeatability of homogenized finite element analysis based on multiple HR-pQCT sections for assessment of distal radius and tibia	strength
			\newblock \textit{Bone} (141)
			
			\bibitem[Whittier et al. 2020]{p5} Whittier et al. (2020)
			\newblock Guidelines for the assessment of bone density and microarchitecture in vivo using high-resolution peripheral quantitative computed tomography
			\newblock \textit{Osteoporosis International} (31), 1607-1627
			
		\end{thebibliography}
	}
\end{frame}



\begin{frame}
	\frametitle{Region of Interest Analysis}
	\footnotesize{
		\begin{thebibliography}{99}
			\setbeamertemplate{bibliography item}[triangle]
			
			\bibitem[Daszkiewicz et al. 2017]{p6} Daszkiewicz, K., Maquer, G., Zysset, P.K. (2017)
			\newblock The effective elastic properties of human trabecular bone may be approximated using micro-finite element analyses of embedded volume elements
			\newblock \textit{Biomechanics and Modeling in Mechanobiology} (16), 731-742
			
			\bibitem[Panyasantisuk et al. 2015]{p7} Panyasantisuk, J., Pahr, D.H., Gross, T., Zysset, P.K. (2015)
			\newblock Comparison of Mixed and Kinematic Uniform Boundary Conditions in Homogenized Elasticity of Femoral Trabecular Bone Using Microfinite Element Analyses
			\newblock \textit{Journal of Biomechanical Engineering} (137), 1-7
			
			\bibitem[Zysset 1995]{p8}  Zysset, P.K., Curnier, A. (1995)
			\newblock An alternative model for anisotropic
			elasticity based on fabric tensors
			\newblock \textit{Mechanics of Materials} (21), 243-250
			
		\end{thebibliography}
	}
\end{frame}



\begin{frame}
	\frametitle{Morphometric Analysis}
	\footnotesize{
		\begin{thebibliography}{99}
			\setbeamertemplate{bibliography item}[triangle]
			
			\bibitem[Folkestad et al. 2012]{p9} Folkestad, L., Hald, J.D., Hansen, S., Gram, J., Langdahl, B., Abra-
			hamsen, B., Brixen, K. (2012)
			\newblock Bone geometry, density, and microarchitecture in the distal radius and tibia in adults with osteogenesis imperfecta type i assessed by high-resolution pQCT
			\newblock \textit{Journal of Bone and Mineral Research} (27), 1405-1412
			
			\bibitem[Kocijan et al. 2015]{p10} Kocijan, R., Muschitz, C., Haschka, J., Hans, D., Nia, A., Geroldinger, A., Ardelt, M., Wakolbinger, R., Resch, H. (2015)
			\newblock Bone structure assessed by HR-pQCT, TBS and DXL in adult patients with different types of osteogenesis imperfecta
			\newblock \textit{Osteoporosis International
			} (26), 2431-2440
			
			\bibitem[Rolvien et al. 2018]{p11} Rolvien, T., Stürznickel, J., Schmidt, F.N., Butscheidt, S., Schmidt, T., Busse, B., Mundlos, S., Schinke, T., Kornak, U., Amling, M., Oheim, R. (2018)
			\newblock Comparison of Bone Microarchitecture Between
			Adult Osteogenesis Imperfecta and Early-Onset Osteoporosis
			\newblock \textit{Calcified Tissue International} (103), 512-521
			
		\end{thebibliography}
	}
\end{frame}



\begin{frame}
	\frametitle{Other References}
	\footnotesize{
		\begin{thebibliography}{99}
			\setbeamertemplate{bibliography item}[triangle]
			
			\bibitem[Indermaur et al. 2021]{p12}  Indermaur, M. et al. (2021)
			\newblock Compressive Strength of Iliac Bone ECM Is Not Reduced in Osteogenesis Imperfecta and Increases With Mineralization
			\newblock \textit{Journal of Bone and Mineral Research}
			
			\bibitem[Gross et al. 2013]{p13}  Gross, T., Pahr, D.H., Zysset, P.K. (2013)
			\newblock Morphology elasticity relationships using decreasing fabric information of human trabecular bone from three major anatomical locations
			\newblock \textit{ Biomechanics and Modeling in Mechanobiology}, 793–800
			
		\end{thebibliography}
	}
\end{frame}

%----------------------------------------------------------------
%----------------------------------------------------------------
%----------------------------------------------------------------

\end{document} 